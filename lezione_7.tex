\subsection{Example of vector space with infinite dimesnion}
Let $\mathbb{R}\left[x\right]$ be the set of polynomials with coefficents in $\mathbb{R}$
$$\mathbb{R}\left[x\right] = \left\{a_0 + a_1 x + a_2 x^2 + a_3 x^3 + \dots + a_n x^n : a_0, \dots, a_n \in \mathbb{R} \quad \forall n \in \mathbb{N}\right\}$$
Given $P(x) = a_0 + a_1 x + \dots + a_n x^n$ where $a_n \not = 0$ and we say that the degree $\deg P(x)$ is $n$
\begin{example}
    $$P_1(x) = 1 + 2x + 3x^3 + x^4 \qquad P_2(x)= \frac{1}{2} - x + 2x^2$$
    $$P_1(x) + P_2(x) = \frac{3}{2} + x + 2x^2 + 3x^3 + x^4$$
\end{example}
\begin{example}
    $$P(x) = -3 + x + \sqrt{2}x^2 + x^5 -7x^6, \qquad k = 2$$
    $$kP(x) = -6 + 2x + 2\sqrt{2}x^2 + 2x^5 - 14x^6$$
\end{example}
The set $\mathbb{R}\left[x\right]$ with these operations is a vector space.

There is a one to one corresponcence (an injective and surjective map) between $\mathbb{R}\left[x\right]$ and $\mathbb{R}^\infty$ what is the set of vectors of infinite lenght
\begin{align*}
   f: \mathbb{R}\left[x\right] &\to \mathbb{R}^\infty \\
   P(x) &\mapsto (a_0, a_1, \dots, a_n, 0,0,0, \dots)
\end{align*}
This map is also a linear map (meaning that $f(P_1(x) + P_2(x)) = f(P_1(x)) + f(P_2(x)),\ f(kP(x)) = kf(P(x)))$

This $f$ is an isomorphian and $\mathbb{R}\left[x\right]$ is «equal» (up to isomorphian) to $\mathbb{R}^\infty$.

A basis of $\mathbb{R}^\infty$ is
$$\left\{(1,0,0, \dots), (0,1,0, \dots), (0,0,1,0,0, \dots), \dots \right\}$$
and the dimension of $\mathbb{R}^\infty$ is $\infty$
$$P(x) = a_0 \cdot 1 + a_1 \cdot x + a_2 \cdot x^2 + \dots + a_n \cdot x^n + 0 \cdot x^{n+1} + \dots$$
\section{Determinant of a matrix}
Consider $A\vec{x} = \vec{b}$ for solving it we transorm using elementary operations the augmented matrix $\left(A|\vec{b}\right)$ in RREF.

If $A \in \mathbb{R}^{n \times n}$, we can solve $A\vec{x} = \vec{b}$ also in another way
\begin{align*}
    5x &= 3\\
    x &= \frac{1}{5} \cdot 3
\end{align*}
if $A \in \mathbb{R}^{n\times n}$ is an invertible matrix, that is there exists a matrix $B \in \mathbb{R}^{n \times n}$ such that $AB = BA = I_n$ and we denote $B$ by $A^{-1}$ and we call it the inverse of $A$. This if $A$ is invertible and the inverse in $A^{-1}$ then 
$$A\vec{x} = \vec{b} \Leftrightarrow A^{-1} \cdot A \vec{x} = A^{-1} \vec{b} \Leftrightarrow I_n \cdot \vec{x} = A^{-1}\vec{b} \Leftrightarrow \vec{x} = A^{-1} \vec{b}$$
For studying the invertibility of a square matrix we introduce the notion of determinat.

The determinat is a real number associated to a square matrix. Let us start with a matrix $A \in \mathbb{R}^{2 \times 2}$ which is
$$A = \begin{bmatrix}
    a & b \\
    c & d
\end{bmatrix} \qquad a,b,c,d \in \mathbb{R}$$
$$\det(A) = ad-bc$$
Consider now $A \in \mathbb{R}^{3\times 3}$ which is
$$A = \begin{bmatrix}
    a_{11} & a_{12} & a_{13} \\
    a_{21} & a_{22} & a_{23} \\
    a_{31} & a_{32} & a_{33}
\end{bmatrix}$$
$$\det(A) = a_{11} \cdot \det \begin{bmatrix}
    a_{22} & a_{23} \\
    a_{32} & a_{33}
\end{bmatrix} - a_{12} \cdot \det \begin{bmatrix}
    a_{21} & a_{23} \\
    a_{31} & a_{33}
\end{bmatrix} + a_{13} \cdot \det \begin{bmatrix}
    a_{21} & a_{22} \\
    a_{31} & a_{32}
\end{bmatrix}$$
In general given $A \in \mathbb{R}^{n \times n}$, $A=(a_{ij})$
$$\det A = a_{11} \cdot \det(A_{11}) - a_{12} \cdot \det(A_{12}) + a_{13} \cdot \det(A_{13}) - a_{14} \cdot \det(A_{14}) + \cdots + (-1)^n a_{1n}\det(A_{1n})$$

where $A_{11} \in \mathbb{R}^{n-1 \times n-1}$ obtained from $A$ removing first row and first column

where $A_{12} \in \mathbb{R}^{n-1 \times n-1}$ obtained from $A$ removing first row and second column

where $A_{1j}\in \mathbb{R}^{n-1 \times n-1}$ obtained from $A$ removing first row and $j^{th}$ column for $j = 1, \dots, n$

This is the so-called Laplace expansion of the determinat.
\begin{remark}[Remark]
    If $A\in \mathbb{R}^{2 \times 2}$, then $\det(A)$ is the area of the parallelogram with sides the columns of $A$.

    If $A \in \mathbb{R}^{3 \times 3}$, then $\det(A)$ is the volume of the parallelepiped with sides the columns of $A$. Similarly this works if $A \in \mathbb{R}^{n \times n}$, $\det(A)$ is the volume of the parallelepiped with dimension $n$ and sides the columns of $A$.
\end{remark}
\begin{remark}[Remark]
    A matrix $A\in \mathbb{R}^{n \times n}$ is invertible if and only if $\det(A) \not = 0$
\end{remark}
\begin{remark}[Remark]
    Consider $A \vec{x} = \vec{b}$ $A \in \mathbb{R}^{n \times n}$, the solution is unique if and only if $\det(A) \not = 0$ and the solution is given by $x_i = \frac{\det(A_i)}{\det(A)}$ where $A_i \in \mathbb{R}^{n \times n}$ obtained from $A$ replacing the $i^{th}$ column of $A$ with $\vec{b}$

    This is the Cramer rule.
\end{remark}
\begin{example}
    $$A = \begin{bmatrix}
        1 & 0 & 2 \\
        0 & 2 & 4 \\
        0 & 0 & 7 
    \end{bmatrix} \qquad B = \begin{bmatrix}
        0 & 1 & 2 \\
        1 & 1 & 4 \\
        1 & 0 & 0
    \end{bmatrix}$$
    \begin{align*}
        \det(A) &= 1 \cdot \det \begin{bmatrix}
            2 & 4 \\
            0 & 7
        \end{bmatrix} - 0 \det \begin{bmatrix}
            0 & 4 \\
            0 & 7
        \end{bmatrix} + 2 \det \begin{bmatrix}
            0 & 2 \\
            0 & 0
        \end{bmatrix} \\
        &= 1 \times 14 - 0 + 2 \times 0 = 14 \\
        \det(B) &= -1 \cdot \det \begin{bmatrix}
            1 & 4 \\
            1 & 0
        \end{bmatrix} + 2 \det \begin{bmatrix}
            1 & 1 \\
            1 & 0
        \end{bmatrix} = 4 - 2 = 2
    \end{align*}
\end{example}
\begin{remark}[Remark]
    For computing the determinat we can apply the definition exploiting any row (not necessarly the firt one).  

    This, if a matrix has a row with many zeros is more convenient to use.
\end{remark}
\begin{example}
    $$B = \begin{bmatrix}
        0 & 1 & 2 \\
        1 & 1 & 4 \\
        1 & 0 & 0
    \end{bmatrix}$$
    $$\det(B) = 1 \cdot \det \begin{bmatrix}
        1 & 2 \\
        1 & 4
    \end{bmatrix} + 0 + 0 = 2$$
    $$C = \begin{bmatrix}
        1 & -1 & 2 & 4 \\
        0 & 0 & 0 & 0 \\
        -7 & 5 & 1 & 2 \\
        3 & 1 & -5 & 1 
    \end{bmatrix} \ \det(C) = 0$$
\end{example}
\subsection{Properties}
\begin{enumerate}
    \item If $A \in \mathbb{R}^{n \times n}$ is in REF and the number of pivots is $n$, then $\det(A)$ is equal to the product of pivots
    \item If $A \in \mathbb{R}^{n \times n}$ is in REF and the number of pivots is $<n$, then $\det(A) = 0$
    \item Given $A \in \mathbb{R}^{n\times n}$ and let $B$ be a matrix obtained from $A$ switching two rows, then $\det(B) = - \det(A) \ (\det(A) = - \det(B))$
    
    $$A = \begin{bmatrix}
        1 & 0 & 2 \\
        0 & 2 & 4 \\
        0 & 0 & 7
    \end{bmatrix} \qquad B = \begin{bmatrix}
        0 & 2 & 4 \\
        1 & 0 & 2 \\
        0 & 0 & 7 
    \end{bmatrix}$$

    check that $\det(A) = - \det(B)$
    \item Given $A \in \mathbb{R}^{n \times n}$ and let $B$ be a matrix obtained from $A$ multiplying a row of $A$ by a scalar $k$, then $\det(B) = k \det(A)$ 
    
    $$A = \begin{bmatrix}
        1 & 0 & 2 \\
        0 & 2 & 4 \\
        0 & 0 & 7
    \end{bmatrix} \qquad B = \begin{bmatrix}
        1 & 0 & 2 \\
        0 & 2 & 4 \\
        0 & 0 & 1
    \end{bmatrix}$$

    check that $\det(B) = \frac{1}{7} \cdot \det(A)$
    \item Given $A \in \mathbb{R}^{n \times n}$ and let $B$ be a matrix obtained from $A$ adding to a row of $A$ a multiple of another row, then $\det(B) = \det(A)$
    
    $$A = \begin{bmatrix}
        1 & 0 & 2 \\
        0 & 2 & 4 \\
        0 & 0 & 7
    \end{bmatrix} \qquad B = \begin{bmatrix}
        1 & 1 & 4 \\
        0 & 2 & 4 \\
        0 & 0 & 7
    \end{bmatrix}$$

    check that $\det(B) = \det(A)$ 
\end{enumerate}