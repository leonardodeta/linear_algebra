\begin{remark}[Remark]
    Pay attention! If you have a matrix $A$ ans transorm it replacing, e.g., the first row $R_1$ with the row $2R_1 + 3R_2$ then we obrain a new matrix $B$ and $\det B = 2 \det A$
\end{remark}
\subsection{Proprieties}
Given $A, B \in \mathbb{R}^{n\times n}$, $\det(AB) = \det(A) \det (B)$ let us see a method for computing the inverse of a matrix $A \in \mathbb{R}^{n \times n}$ such that $\det A \not = 0$.

We want to compute an unknown matrix $A^{-1} \in \mathbb{R}^{n\times n}$ such that 
$$A \cdot A^{-1} = I_n$$
that is
$$\begin{bmatrix}
    a_{11} &\cdots &a_{1n} \\
    \vdots &\ddots &\vdots \\
    a_{n1} &\cdots &a_{nn}
\end{bmatrix} \begin{bmatrix}
    x_{11} &\cdots &x_{1n}\\
    \vdots &\ddots & \vdots\\
    x_{n1} &\cdots &x_{nn}
\end{bmatrix} = \begin{bmatrix}
    1 &\cdots &0\\
    \vdots &\ddots & \vdots\\
    0 &\cdots & 1
\end{bmatrix}$$
$a_{ij}$ are known real numbers

$x_{ij}$ are unknown real numbers
$$\begin{cases}
    a_{11}x_{11} + \cdots \ + a_{1n}x_{n1} =1\\
    a_{21}x_{11} + \cdots \ + a_{2n}x_{n1} =0\\
    \qquad \cdots\\
    a_{n1}x_{11} + \cdots \ + a_{nn}x_{n1} =0
\end{cases} \qquad \text{Linear system with $n$ unknown and $n$ equations}$$
$$\Updownarrow$$
\begin{center}
    Linear system whose augmented matrix is $\left(A | \vec{e}_1 \right)$ where $\vec{e}_1 = (1,0,\dots, 0)$
\end{center}
Similarly, if we focus on the seond column of $A^{-1}$, we get a linear system whose unknone are $(x_{12}, x_{22}, x_{32}, \dots, x_{n2})$ and the augmented matrix is 
$$\left(A | \vec{e}_2 \right) \qquad \text{where} \qquad \vec{e}_2 = (0,1,0,\dots, 0)$$
This, in order to find $A^{-1}$ we need to solve $n$ linear system whose augmented matrices are
$$\left(A | \vec{e}_1 \right), \left(A | \vec{e}_2 \right), \dots, \left(A | \vec{e}_n \right)$$
We can solve all these linear systems considering
$$\left(A | I_n \right)\quad \leftarrow \quad \det A \not = 0$$
and transforming it into RREF we obtain 
$$\left(I_n | A^{-1}\right)$$
\begin{example}
    $$A = \begin{bmatrix}
        1 &2 \\
        3 &0
    \end{bmatrix} \qquad \det A = -6 \not = 0 \Rightarrow A \text{ inverible}$$
    $$\left[\begin{array}{cc|cc}
        1 &2 &1 &0\\
        3 &0 &0 &1\\
    \end{array}\right] \longrightarrow \left[ \begin{array}{cc|cc}
        1 &2 &1 &0 \\
        0 &6 &3 &-1        
    \end{array}\right] \begin{array}{l}
        R_1\\
        3R_1 - R_2        
    \end{array}$$
    $$\longrightarrow \left[ \begin{array}{cc|cc}
        3 &0 &0 &1 \\
        0 &6 &3 &-1
    \end{array} \right] \begin{array}{l}
        3R_1 - R_2\\
        R_2
    \end{array} \longrightarrow \left[\begin{array}{cc|cc}
        1 &0 &0 &\frac{1}{3}\\
        0 &1 &\frac{1}{2} &-\frac{1}{6}
    \end{array}\right] \begin{array}{l}
        \frac{1}{3}R_1\\
        \frac{1}{6}R_2
    \end{array} \quad \text{RREF}$$
    $$A^{-1} = \begin{bmatrix}
        0 &\frac{1}{3}\\
        \frac{1}{2} &-\frac{1}{6}
    \end{bmatrix}$$
    $$A \cdot A^{-1} = \begin{bmatrix}
        1 &2 \\
        3 &0
    \end{bmatrix} \begin{bmatrix}
        0 &\frac{1}{3}\\
        \frac{1}{2} &-\frac{1}{6}
    \end{bmatrix} = \begin{bmatrix}
        1 &0 \\
        0 &1
    \end{bmatrix} $$
    Check that $A^{-1} \cdot A = \begin{bmatrix}
        1 &0\\
        0 &1
    \end{bmatrix}$
\end{example}
\section{Linear function}
\begin{definition}
    Given two vector spaces $V$ and $W$, with finite dimension. $V \simeq \mathbb{R}^n$ and $W \simeq \mathbb{R}^m$ a linear function $f : \mathbb{R}^n \to \mathbb{R}^m$ is a map such that
    \begin{enumerate}
        \item $\forall \ \vec{v}_1, \vec{v}_2 \in \mathbb{R}^n, f(\vec{v}_1 + \vec{v}_2) = f(\vec{v}_1) + f(\vec{v}_2)$
        \item $\forall \ \vec{v} \in \mathbb{R}^n, \forall \ k \in \mathbb{R}, f(k \vec{v}) = k f(\vec{v})$
    \end{enumerate}
    $$f\left(\vec{0}\right) = \vec{0}$$
    $$ \vec{0} \in \mathbb{R}^n \qquad \vec{0} \in \mathbb{R}^m$$
\end{definition}
By 1. we know that $\forall \ \vec{v} \in \mathbb{R}^n$, then $f\left(\vec{0} + \vec{v}\right) = f\left(\vec{0}\right) + f\left(\vec{v}\right) = f\left(\vec{v}\right) \Rightarrow f(\vec{v}) = f\left(\vec{0}\right) + f\left(\vec{v}\right) \Rightarrow f\left(\vec{0}\right) = \vec{0}$ 

If we know the image of the vectors of a basis of $\mathbb{R}^n$, then we are able to compute $f$ an all the vectors of $\mathbb{R}^n$.

If $B = \left\{\vec{b}_1, \dots, \vec{b}_n \right\} \subseteq \mathbb{R}^n$ is a basis of $\mathbb{R}^n$ then, $\forall \ \vec{v} \in \mathbb{R}^n$, we have that $\vec{v} = x_1 \vec{b}_1 + x_2 \vec{b}_2 + \dots + x_n \vec{b}_n$ fore some $x_1, \dots, x_n \in \mathbb{R}$ and
\begin{align*}
    f(\vec{v}) &= f\left(x_1 \vec{b}_1 + x_2 \vec{b}_2 + \dots + x_n \vec{b}_n\right)\\
    &= x_1 f\left(\vec{b}_1\right) + x_2 f\left(\vec{b}_2\right) + \dots + x_n f\left(\vec{b}_n\right)
\end{align*}
Let $C = \left\{\vec{c}_1, \dots, \vec{c}_m \right\} \subseteq \mathbb{R}^m$ be a basis of $\mathbb{R}^m$, if we know that 
$$f\left(\vec{b}_1 \right) = \vec{w}_1 = a_{11}\vec{c}_1 + a_{21}\vec{c}_2 + \dots + a_{m1}\vec{c}_m $$
$$f\left(\vec{b}_2 \right) = \vec{w}_2 = a_{12}\vec{c}_1 + a_{22}\vec{c}_2 + \dots + a_{m2}\vec{c}_m $$
$$\cdots$$
$$f\left(\vec{b}_n \right) = \vec{w}_n = a_{1n}\vec{c}_1 + a_{2n}\vec{c}_2 + \dots + a_{mn}\vec{c}_m $$
We define
$$A = \left[f\left(\vec{b}_1\right) \ f\left(\vec{b}_2\right) \ \cdots \ f\left(\vec{b}_n\right) \right]$$
$$A = \begin{bmatrix}
    a_{11} &a_{12} &\cdots &a_{1n}\\
    a_{21} &a_{22} &\cdots &a_{2n}\\
    \vdots &\vdots &\ddots &\vdots\\
    a_{m1} &a_{m2} &\cdots &a_{mn}
\end{bmatrix} \in \mathbb{R}^{m \times n}$$
This is the matrix whose columns are the component of $f(\vec{b}_1)$ w.r.t. the basis $C$ of $\mathbb{R}^m$. This matrix $A$ is the matrix associated to $f$ and «it works like $f$»

Take $\vec{v} \in \mathbb{R}^n$, $\vec{v} = v_1 \vec{b}_1 + \cdots + v_n\vec{b}_n$
\begin{align*}
    f(\vec{v}) &= v_1 f\left(\vec{b}\right)_1 + \cdots + v_n f\left(\vec{b}_n\right) = v_1 \vec{w}_1 + \cdots + v_n \vec{w}_n = \\
    &= v_1\left(a_{11} \vec{c}_1 + \dots + a_{m1} \vec{c}_m \right) + \cdots + v_n \left(a_{1n} \vec{c}_1 + \cdots + a_{mn} \vec{c}_m \right) \\
    &= \left( v_1 a_{11} + \cdots + v_n a_{1n} \right) \vec{c}_1 + \cdots + \left( v_1 a_{m1} + \dots + v_n a_{mn} \right) \vec{c}_m \\
    &= W \in \mathbb{R}^m
\end{align*}
where $\vec{w} = A\vec{v} \Leftrightarrow \vec{w} = f\left(\vec{v}\right)$
\begin{align*}
    \begin{bmatrix}
        a_{11} & \cdots & a_{1n} \\
        \vdots & \ddots & \vdots \\
        a_{m1} & \cdots & a_{mn}
    \end{bmatrix}
    \begin{bmatrix}
        v_1 \\
        \vdots \\
        v_n
    \end{bmatrix}
    &=
    \begin{bmatrix}
        a_{11}v_1 + \cdots + a_{1n}v_n \\
        \vdots \\
        a_{m1}v_1 + \cdots + a_{mn}v_n
    \end{bmatrix}
    \\[6pt]
    \mathbb{R}^{m \times n}\quad
    \mathbb{R}^{n \times 1}
    &\qquad
    \mathbb{R}^{m \times 1}
\end{align*}



