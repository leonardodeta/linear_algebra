\documentclass[12pt]{article}

% ===== Pacchetti base =====
\usepackage[utf8]{inputenc}
\usepackage[english]{babel}
\usepackage{geometry}
\geometry{margin=2.5cm}

% ===== Pacchetti matematici =====
\usepackage{amssymb}
\usepackage{amsmath}

% ===== Colori, link e grafica =====
\usepackage{xcolor}
\usepackage{graphicx}
\usepackage{hyperref}
\hypersetup{
    hidelinks % Nasconde i link
}
\usepackage[most]{tcolorbox}  % <-- aggiunto [most] per usare opzioni avanzate
\usepackage{titling}
\usepackage{setspace}

% ===== Impostazioni per i listati =====
\usepackage{listings}
\lstset{
    basicstyle=\ttfamily,
    breaklines=true,
    breakatwhitespace=true,
    postbreak=\mbox{\textcolor{red}{$\hookrightarrow$}\space}
}

% ===== Ambienti personalizzati =====
\newtcolorbox{definition}{
  colback=green!5!white,
  colframe=green!40!black,
  title=Definition
}

\newtcolorbox{example}{
  colback = yellow!5!white,
  colframe = yellow!40!black,
  title = Example
}

\newtcolorbox{formulabox}[1][]{
  colback=blue!5!white,
  colframe=blue!60!black,
  fonttitle=\bfseries,
  title=#1,
  boxrule=0.8pt,
  arc=3mm,
  left=6pt, right=6pt, top=6pt, bottom=6pt
}

\newtcolorbox{remark}[1][]{
  colback=gray!5!white,
  colframe=gray!60!black,
  fonttitle=\bfseries,
  title=#1,
  boxrule=0.8pt,
  arc=3mm,
  left=6pt, right=6pt, top=6pt, bottom=6pt,
  enhanced,
  colbacktitle=gray!50!white
}

\newtcolorbox{theorem}[1][]{
  colback=red!5!white,
  colframe=red!60!black,
  fonttitle=\bfseries,
  title=#1,
  boxrule=0.8pt,
  arc=3mm,
  left=6pt, right=6pt, top=6pt, bottom=6pt,
  enhanced,
  colbacktitle=red!50!white
}

% ===== Stile del titolo =====
\pretitle{\begin{center}\Huge\bfseries}
\posttitle{\par\vspace{0.5em}\hrule height 1pt \vspace{0.5em}\end{center}}
\preauthor{\begin{center}\large}
\postauthor{\end{center}}
\predate{\begin{center}\normalsize}
\postdate{\end{center}}

\title{\textcolor{blue!50!black}{Linear algebra}}
\author{
    \textbf{Leonardo Detassis} \\[6pt]
    \small Master's Degree in Data Science \\[2pt]
    \small University of Trento – Department of Mathematics
}
\date{2025}

\begin{document}

\maketitle
\newpage
\tableofcontents

\newpage

\section{Introduction}
In linear algebra we study object, also a point is a linear object. Usually we study this object from an geometry point of view, in this course we see the algebrian point of view. When we do \textbf{operations} we are doing algebra. Vector are a kind of linear object, we can see a vector like a point.

In $\mathbb{R}^3$ we have more kind of linear objects. In $\mathbb{R}^4$ we start to have problem whit the geometrical point of view. Whit equations i can describe objects in every dimension. Vector spaces is very important in linear algebra. Matrix are very important because we also have functions for transform object.
\section{Sets}
We denote by $\emptyset$ the empty set.
\begin{align*}
&\mathbb{N} = \left\{ 0, 1, 2, \cdots \right\} \text{ Natural numbers}\\
&\mathbb{Z} = \left\{\cdots, -2, -1, 0, 1, 2, \cdots \right\} \text{ Integers}\\
&\mathbb{Q} = \left\{ \frac{a}{b} : a, b \in \mathbb{Z}, b \ne 0 \right\}
\end{align*}
\begin{center}
    $\mathbb{R}$ the set of all  real numbers, Every real number is the limit of a sequence of rational numbers.
\end{center}

\subsection{Examples}
\begin{align*}
&A = \left\{ x \in \mathbb{R} : -5 \le x \le 2\right\}11
&B = \left\{ x \in \mathbb{Z} : -5 \le x \le 2 \right\}\\
&C = \left\{ x \in \mathbb{R} : x > 1 \right\}
\end{align*}
\subsection{Operation between sets}
Given two set $A$ and $B$ we define the following operation: 
\begin{itemize}
    \item Intersection: $A \cap B = \left\{ x : x \in A \cap x \in B\right\}$
    \item Union $A \cup B = \left\{ x: x \in A \lor X \in B \right\}$
    \item Difference $A \setminus B = \left\{ x : x \in A \land x \not\in B \right\}$
    \item Cartesian product $A \times B = \left\{ (x,y): x \in A \land y \in B \right\}$
\end{itemize}
\section{Cartesian plane}
The cartesian plane is a 2 dimensional object. All the $\mathbb{R}$ can be represented on an «oriented line» where we fixed a special point, namly the origin O. In a similar way we can describe all the elements of $\mathbb{R} \times \mathbb{R} = \mathbb{R}^2 = \left\{ (x, y) : x \in \mathbb{R}, y \in \mathbb{R}\right\}$ with the cartesian plane. We take 2 oriented lines which are orthogonal and we call the origin O the interaction between these lines (representing the element $(0, 0) \in \mathbb{R}^2$.

The horizontal line is called the axis of abscises or the x-axis. The vertical line is called of ordinate or the y-axis.

Evrey point $P$ descibe an element of $\mathbb{R}^2$ since it is identified by two set number $X_P$ and $X_P$ which are its cordinated, i. e., 

\begin{align*}
    \text{The y-axis is the following set } &= \left\{ (0,y) \in \mathbb{R}^2 \right\} = \left\{ (0,y): y \in \mathbb{R}\right\} \\
    & = \left\{ (x,y) \in \mathbb{R}^2 : x = 0 \land y \in \mathbb{R} \right\}
\end{align*}
\begin{align*}
    \text{The x-axis is the following set } &= \left\{ (x,0) \in \mathbb{R}^2 \right\} 
    = \left\{ (z,0): z \in \mathbb{R}\right\} \\
    &= \left\{ (x_1,x_2) \in \mathbb{R}^2 : x_2 = 0 \right\}
\end{align*}

Given two point $A$ and $B$ $\in \mathbb{R}^2$, we can evaluate theri distance (euclidean distance) which is the lengh of the segment $\overline{AB}$:
\begin{align*}
    d(A,B)=||\overline{AB}|| &= \sqrt{(x_B - x_A)^2 + (y_B- y_A)^2} \\
    &= \sqrt{(x_A - x_B)^2 + (y_A- y_B)^2}
\end{align*}
\subsection{Vector in the cartesian plane}
In general a vector is an arrow. Given a point $P$ in the cartesian plane, i. e., a point of two numbers $(x_P, y_P) \in \mathbb{R}^2$ it also indentified a vector of the plane. 
The point $P$ identifies the vector $\vec{OP} = (x_P, y_p)$. Sometimes we will also use the notion $\vec{\mathbf{v}}$ or $\mathbf{v}$ for a vector.
$$\mathbf{v} = (\mathbf{v}_1, \mathbf{v}_2) \in \mathbb{R}^2$$
We have that, from this algebric point of view, point and vector are equivalent, they are the same object described by a point or set. 
\subsection{Operations in the cartesian plane}
\subsubsection{Scalar multiplication}
Multiplication by a sclar, i. e., by a real number. Givern a point $A = (x_A, y_A)(\Leftrightarrow Q = (x_A, y_A) = \vec{OA})$ and a scalr $t\in \mathbb{R}$ we define the multiplication by a scalar $tA = (tx_A, ty_A)$. 
$$\mathbb{R}^2 \times\mathbb{R} \longrightarrow \mathbb{R}^2$$
$$(A\times t) \longrightarrow tA$$

The effect of the multiplication by a scalar on a vector $\vec{\textbf{a}}$ ub fact the vertor $\mathbb{a}$ is stretched

\subsubsection{Sum}
Given two points $A = (x_A, y_A)( \Leftrightarrow a = (x_A, y_A))$ and $B= (x_B,y_B)$ the sum is defined as
$$A+B = (x_A+x_B, y_A+y_B)$$
\begin{align*}
\mathbb{R}^2 \times \mathbb{R}^2 &\longrightarrow \mathbb{R}^2\\
(A,B) &\longrightarrow A+B
\end{align*}

$a + b (\Leftrightarrow A+B)$ is the diagonal of the parallelogram whith $a$ and $b$ 

the following proprieties hold:
$$\forall \ a, b, c \in \mathbb{R}^2, \forall \ t \in \mathbb{R} \text{ we have}$$
\begin{itemize}
    \item $a+b = b+a$
    \item $(a+b) + c = a + (b+ c)$
    \item $t(a+b)=ta + ta$
    \item $a + 0 = a$ when $0 = (0,0) =$ origin
    \item $(-1)a = -a$ when $a+(-a)=0$
\end{itemize}
\subsection{Lines in the cartesian plane}
Given a point $A = (x_A, y_A) \in \mathbb{R}^2$, there exist one and only one line $\ell$ through the origin O and the point A. Now we went to describe the coordinates of all point. 

Let us take a generic point $P=(x_P, y_P)\in \ell$ and let us find its coordinate in term of $A$. The triangles $OAH$ and $OPC$ are similar (they have the same angles) and consequently the side are proportion (by the same factors):
$$||\overline{OK}|| = t||\overline{OH}|| \quad \text{and} \quad  ||\overline{PK}|| = t ||\overline{AH}|| \text{ for a certain } t \in \mathbb{R}$$
that is
$$\begin{cases}
    x_p' =t x_A\\
    y_p' = ty_A
\end{cases} \quad \forall \ t \in \mathbb{R} \qquad \begin{cases}
    x = tx_a\\
    y= ty_A
\end{cases} \quad \forall \ t \in \mathbb{R}$$
These are the parametric equations of a line through the origin and given a point $A$. 
$$A = (2,3) \qquad \begin{cases}
     x= 2t \\
    y = 3t
\end{cases} \quad \forall \ t \in\mathbb{R}$$
$$\text{In other words} \quad \ell = \left\{ (x,y) \in \mathbb{R}^2: x=tx_A, y=ty_A, \forall \ t \in \mathbb{R} \right\} \quad (1)$$
The cartesian equation of a line through the origin is 
$$y = mx \qquad \text{where} \qquad m = \frac{y_A}{x_A} \quad \text{slope}$$
$$\ell \left\{ (x,y) \in \mathbb{R}^2 : y = mx, m=\frac{y_A}{x_A} \right\} \quad (2)$$
Let us Considering the set $(1)$ and check that $O$ ans $A$ belong to this set:
$$O \in \ell \text{ because if we take } t=0, \text{ we get } x = 0 \cdot X_A= 0, y = 0\cdot y_A = 0 
$$
$$\Rightarrow (0,0) \in \ell$$
$A \in \ell$ because if we take $t=1$, we have $x=x_A$, $y=y_A \Rightarrow A \in \ell$

Let us check that $O$ and $A$ belong to $   \ell$ exploiting the set defined in $(2)$:
\begin{align*}
    &O = (0,0) \in \ell \quad\text{ is true, because } \quad 0 = m\cdot 0\\
    &A= (x_A, y_A) \in \ell \quad \text{ is true, because } \quad y_A = mx_A \Leftrightarrow y_A = \frac{y_A}{\not{ x_A}} \cdot \not x_A
\end{align*}
We can check that a point provided by the parametric equations satisfies the cartesian equation. 

Let us consider a point $(tx_A, ty_A)$ and check if it satisfies the cartesian equation $y=mx$:
$$ty_A = mtx_A \Leftrightarrow ty_A = \frac{y_A}{\not x_A}t \not x_a \Leftrightarrow ty_A = ty_Ax$$
\begin{example}
    Consider the line through the origin and $A = (5,2)$.
    $$\text{The parametric equation are } \quad \begin{cases}
        x= 5t\\
        y= 2t
    \end{cases} \quad \forall t \in \mathbb{R} $$
    $$t = 0 \quad (0,0), \qquad t= 1 \quad (5,2), \qquad t=-2 \quad \left(-10,-4\right),  $$ 
    $$t= \frac{1}{3} \quad \left(\frac{5}{3}, \frac{2}{3}\right) \qquad \dots \qquad t=\sqrt{3} \quad \left(5\sqrt{3},2\sqrt{3}\right)$$
    All belong to the line. 
    
    The point $(10,3)$ belong to the line? $ \qquad t=2 \quad (10,4)$

    In order to answer to this question we must solve the following linear system
    $$\begin{cases}
        5t = 10\\
        2t = 3
    \end{cases}$$
    where $t$ is the unknown, but this linear system doesn't have solutions!

    This $(10,3)$ doesn't belong to the line
    The cartesian equation of this line is $y = \frac{2}{5}x \to 3=\frac{2}{5}\cdot10$ is not true! is not an identity
\end{example}
Now, let us describe the parametric equations of a line (not necessary through the origin). Given two points $A=(x_A, y_A)$ and $B = (x_B, y_B)$ we want to describe the coordinates of a generic point $P$ over the line through $A$ and $B$. We can consider the line $\ell'$ parallel to $\ell$ and passing through the origin. The parametric equations of $\ell'$ are:
$$\begin{cases}
    x = t(x_B-x_A)\\
    y = t(y_B-y_A)
\end{cases} \quad \forall \ t \in \mathbb{R}$$
The parametric equations of $\ell$ are
$$\begin{cases}
    x = x_A + t(x_B - x_A)\\
    y = y_A + t(y_B - y_A)
\end{cases} \quad \forall \ t \in \mathbb{R}$$
this 
$$\ell \left\{ (x,y) \in \mathbb{R}^2 : x = x_A + t (x_B-x_A), y = y_A + t(y_B-y_A),  \ \forall \ t \in \mathbb{R} \right\}$$
Similarly, tha cartesian equation is
$$y = mx + q$$
where $m = \frac{y_B - y_A}{x_B - x_A}$ slope and $q = \frac{y_A x_B - x_Ay_B}{x_B-x_A}$ constant term

\section{Linear independence (linear dependence)}
If two vectos $\vec v_1$ and $\vec v_2$ lie on the same line (throgh the origin) then they are linearly dependent (from a geometrical point of view).

For an algebraic point of view this means that
$$\vec v_1 = t \vec v_2 \quad t \in \mathbb{R}$$
Two vectors are linearly independent if and only if they are not linearly dependent.

In general if $\vec w_1$ and $\vec w_2$ are linearly independent vectors than all the vectors $\vec v \in \mathbb{R}$ can be written as a linear combination of $\vec w_1$ and $\vec w_2$, this means that
$$\forall \ \vec v \in \mathbb{R}^2 \qquad \vec v = a \vec w_1 + b \vec w_2 \qquad \text{fore some } a,b \in \mathbb{R}$$
and we say that  $\left\{ \vec w_1, \vec w_2 \right\}$ is a basis of $\mathbb{R}^2$.

A special basis is $\left\{ \vec e_1, \vec e_2 \right\}$ where $\vec e_1 = (1,0)$, $\vec e_2 = (0,1)$ and it is called tha canonical basis. 

$\left\{ \vec e_1, \vec e_2 \right\}$ is also an orthogonal basis because $\vec e_1$ and $\vec e_2$ ore orthogonal vectors.
\section{Cartesian space}
The cartesian space is $\mathbb{R}^3 = \mathbb{R} \times \mathbb{R} \times \mathbb{R}$ and its elements are triple of real number $(x,y,z) \quad \forall \ x,y,z \in \mathbb{R}$. 

$O$ is a fixed point of the space, called the origin and we also have three orthogonal lines intersecting in the origin. 

A point $A \in \mathbb{R}^3$ is identified by three real numbers $(x_A, y_A, z_A)$ where:
\begin{itemize}
    \item[] $x_a$ is the distance from the plane $Oyz$
    \item[] $y_a$ is the distance from the plane $Oxz$
    \item[] $z_a$ is the distance from the plane $Oxy$
\end{itemize}
Also in the cartesian space, we can compute the Euclidean distance between the origin $||\overline{OA}|| = d(O,A)$.
Similarly the distance between $A = (x_A, y_A, z_A)$ and $B = (x_B, y_B, z_B)$ is $||\overline{AB}|| = d(A,B) = \sqrt{(x_B-x_A)^2+(y_B-y_A)^2+(z_B-z_A)^2}$.

Similarly to $\mathbb{R}^2$, also in $\mathbb{R}^3$ the vectors are arrows starting from the origin and they are describes by triples of real numbers. 

\begin{remark}[Operations in $\mathbb{R}^3$]
Let us define some operations in $\mathbb{R}^3$:
\begin{itemize}
    \item Scalar multiplication:
    \begin{align*}
    &\mathbb{R} \times \mathbb{R}^3 \longrightarrow \mathbb{R}^3 \\
    &(t, \vec v) \longmapsto t \vec v = (tv_1, tv_2, tv_3)\\
    &(t, V) \longmapsto t V = (tv_1, tv_2, tv_3)
    \end{align*}
    where $\vec v = V = (v_1,v_2,v_3)$
    \item Sum (componentwise):
    \begin{align*}
    &\mathbb{R}^3 \times \mathbb{R}^3 \longrightarrow \mathbb{R}^3\\
    &(\vec v, \vec w) \longmapsto \vec v + \vec w = (v_1+w_1, v_2 + w_2, v_3 + w_3)\\
    &(V, W) \longmapsto V + W = (v_1+w_1, v_2 + w_2, v_3 + w_3)
    \end{align*}
    where $\vec v = (v_1, v_2, v_3) \quad \vec w = (w_1, w_2, w_3)$ 
\end{itemize}
\end{remark}

Given three vectors in $\mathbb{R}^3$, $\vec u, \vec v, \vec w$, we say that they are linearly dependent if there exist three real numbers $a,b,c \in \mathbb{R}$ not all zeros such that
$$a \vec u + b \vec v + c \vec w = 0 = (0,0,0)$$
$$\vec u = \alpha \vec v + \beta \vec w$$
$$a \vec u + b \vec v + c \vec w = 0 \Leftrightarrow a \vec u = - b \vec v - c \vec w$$
If the vectors $\vec u, \vec v, \vec w$, are not linearly dependent, then we sat that they are linearly independent.
In general, given $\vec v_1, \vec v_2, \dots, \vec v_n$ we say that they are linearly dependent if and only if there exist $a_1, a_2, \dots, a_n \in \mathbb{R}$ not all zeros such that
$$a_1 \vec v_1 + a_2 \vec v_2 + \dots + a_n \vec v_n = 0$$
If $\vec v_1, \dots, \vec v_n$ are not linearly dependent, then we say that they are linearly independent. 

In the space, if we have there linearly independent vectors $\vec w_1, \vec w_2, \vec w_3$, then all the vectors $\vec v \in \mathbb{R}^3$ can be written ad a linear combination of $\vec w_1, \vec w_2, \vec w_3$. And we say the $\left\{ \vec w_1, \vec w_2, \vec w_3 \right\}$ is a basis of $\mathbb{R}^3$.

The canonical basis of $\mathbb{R}^3$ is $\left\{ \vec e_1, \vec e_2, \vec e_3 \right\}$ where $\vec e_1 = (1,0,0)$, $\vec e_2 = (0,1,0)$, $\vec e_3 = (0,0,1)$.

Let us check that $\vec e_1, \vec e_2, \vec e_3$ ale linearly independent, we must find if there exist $a,b,c$ not all zeros such that 
$$a   \vec e_1 + b \vec e_2 + v \vec e_3 = \vec 0$$
$$(a,0,0) + (0,b,0) + (0,0,c) = \vec 0$$
$$(a,b,c) = (0,0,0)$$
$$\begin{cases}
    a=0\\
    b = 0\\
    c=0
\end{cases} \iff \vec e_1, \vec e_2, \vec e_3 \text{ are l.i.}$$
$\forall \ \vec v\in \mathbb{R}^3$, $\vec v = (v_1, v_2, v_3)$, then $\vec v = v_1 \vec e_1 + v_2 \vec e_2 + v_3 \vec e_3$. 
\begin{example}
  $$\vec v = (5,3,2) = 5(1,0,0) + 3(0,1,0) + 2(0,0,1)$$
\end{example}
\subsection{Equation of a plane in $\mathbb{R}^3$}
We wold like to characterize all the point belonging to some plane in $\mathbb{R}^3$
\begin{align*}
    \text{The plane}\qquad &Oxy = \left\{ (x,y,0): x,y \in \mathbb{R} \right\} \\
    &Oyz = \left\{ (0,y,z): y,z \in \mathbb{R} \right\} \\
    &Oxz = \left\{ (x,0,z): x,z \in \mathbb{R} \right\} 
\end{align*}
Given a generic plane in the space, it is uniquely determined by: 
\begin{itemize}
    \item Two interacting lines
    \item Three non collinear points 
    \item two linearly independent vectors (identify uniquely a plane through the origin) and a point in $\mathbb{R}^3$
\end{itemize}

We start considering a plane described by two linearly independent vectors, i.e., we are describing all the planes through the origin.

We want to describe any point $P=(x_P, y_P, z_P)$ belonging to the plane identified by two linearly independent vectors $\vec v$ and $\vec w$.

Remember that two linearly independent vectors can be used for constructing all the vectors belonging to the same plane (think, e.g., to the special case of the cartesian plane). All the points $P=(x_P,y_P,z_P)$ belonging to the plane containing the vectors $\vec v$ and $\vec w$ are linear combinations of $\vec v$ and $\vec w$:
$$P= \overline{OP} = (x_P, y_P, z_P) = s \vec v + t \vec w \qquad \forall \ s,t  \in \mathbb{R}$$
The parametric equations of the plane $\Pi$ through the origin and containg  $\vec v$ ans $\vec w$ are
$$\begin{cases}
    x = sv_1 + tw_1\\
    y= sv_2 + tw_2\\
    z = sv_3 + tw_3
\end{cases} \qquad \forall \ s,t \in \mathbb{R}$$
In other words
$$\Pi = \left\{ (x,y,z) \in \mathbb{R}^3 : x = sv_1 + tw_1, y = sv_2+tw_2, z = sv_3+tw_3, \forall \ s,t \in \mathbb{R} \right\}$$
The parametric equations of a plane $\Pi$ identified by two linearly independent vectors $\vec v$ and $\vec w$ and through a point $A = (x_A, y_A, z_A)$ are
$$\begin{cases}
    x = x_A +  sv_1 + tw_1\\
    y= y_A + sv_2 + tw_2\\
    z = z_A +  sv_3 + tw_3
\end{cases} \qquad \forall \ s,t \in \mathbb{R}$$
$$s = 0,\ t= 0 \quad \begin{cases}
    x = x_A \\
    y = y_A \\
    z = z_A
\end{cases}$$
We can obtain the cartesian equation of a plane. If we focus on the equations $\begin{cases}
    x = x_A + sv_1 + tw_1 \\
    y = y_A + sv_2 + tw_2
\end{cases}$ where $s$ and $t$ are unknowns. We can solve this linear system and we obtain the solutions:
$$\begin{cases}
    s = ax+by+f \quad a,b,f \in \mathbb{R}\\
    t = cx+dy+g \quad c,d,g \in \mathbb{R}
\end{cases}$$
Now we substitute these values of $s$ and $t$ in the third equations:
$$z = z_A + sv_3 + tw_3$$
and we obtain something like $z = m+nx+py \quad m,n,p \in \mathbb{R}$ Cartesian equation of a plane in the space 
$$\alpha x + \beta y + \gamma z + \delta = 0$$
Cartesian equation, where $\alpha, \beta, \gamma, \delta \in \mathbb{R}$ depending on $\begin{matrix} &x_A,&y_A, &z_A \\  &v_1,&v_2,&v_3\\ &w_1,&w_2,&w_3 \end{matrix}$
\subsection{Equations of a line in $\mathbb{R}^3$}
A line is always described by two points.

Let us start with a line through  the origin and a point $V = (v_1, v_2, v_3)$, i.e., we are considering the line describes by the vector $\vec v = \overline{OV} = (v_1,v_2,v_3)$ and so all the points in this line $\ell$ are described by
$$(x,y,z) = t \vec v = (tv_1, tv_2, tv_3) \quad \forall \ t \in \mathbb{R} \iff \begin{cases}
    x = tv_1 \\
    y = tv_2 \\
    z = tv_3
\end{cases}$$
A general line in the space is then described by a vector $\vec v$ (giving the direction ) and a point $C=(x_C, y_C, z_C)$ and the parametric equations are:
$$\begin{cases}
    x = x_C + tv_1\\
    y = y_C + tv_2 \\
    z = z_C + tv_3
\end{cases} \qquad \forall \ t \in \mathbb{R}$$
The above description of a line is equivalent to describe it giving two point $A = (x_A, y_A, z_A)$ and $B=(x_B,y_B,z_B)$
$$\begin{cases}
    x = x_A + t(x_B-x_A)\\
    y = y_A + t(y_B-y_A) \\
    z = z_A + t(z_B-z_A)
\end{cases} \iff \begin{cases}
    x = x_B + t(x_B-x_A)\\
    y = y_B + t(y_B-y_A) \\
    z = z_B + t(z_B-z_A)
\end{cases} \quad \forall \ t \in \mathbb{R}$$
A line in the space is identified by the intersection between two (non-parallel) plane: 
$$\begin{cases}\alpha x + \beta y+ \gamma z + \delta= 0 \leftarrow \text{Cartesian equation of a plane}\\
\alpha'x+\beta'y+\gamma'z+ \delta' = 0 \leftarrow \text{Cartesian equation of a plane}
\end{cases} \quad \alpha, \beta,\gamma,\delta, \alpha', \beta', \gamma', \delta' \in \mathbb{R}
$$
\section{Vector space}
\begin{itemize}
    \item 0 - dimensional object $\to \mathbb{R}^0$
    \item 1 - dimensional object $\to \mathbb{R}^1$
    \item 2 - dimensional object $\to \mathbb{R}^2$
    \begin{itemize}
        \item 0 - dimensional object (points)
        \item 1 - dimensional object (lines)
    \end{itemize}
    \item 3 - dimensional object $\to \mathbb{R}^3$
    \begin{itemize}
        \item 0 - dimensional object (points)
        \item 1 - dimensional object (lines)
        \item 2 - dimensional object (planes)
    \end{itemize}
\end{itemize}
Take now 2 cubes move one of them along to a new (the $4^{th}$) direction, connect the vertexes and we get a 4-dimensional object and if we extend it at the infinity in all the 4 direction, then we get  the 4 - dimensional space $\to \mathbb{R}^4$. A $n$-dimensional space is  described by $\mathbb{R}^n$.

In $\mathbb{R}^n$, a $k$-dimensional object (where $0 \le k \le b-1$ is described by parametric equations with $k$ different parameter, it has $k$ degrees of freedom. Given a space of dimension $n$ (i.e., $\mathbb{R}^n$) a k-dimensional object is a subspace of $\mathbb{R}^n$ of dimension $k$. 
$$\mathbb{R}^n = \mathbb{R} \times \mathbb{R} \times \dots \times \mathbb{R}= \left\{ (x_1, x_2, \dots, x_n): x_1, x_2, \dots, x_n \in \mathbb{R} \right\} $$
Similarly to $\mathbb{R}^2$ and $\mathbb{R}^3$, also in $\mathbb{R}^n$ points and vectors are the same objects described by $n$-tuple of real numbers.
\begin{itemize}
    \item Multiplication by a scalar
    \begin{align*}
        &\mathbb{R} \times \mathbb{R}^n \longrightarrow \mathbb{R}^n\\
        &(t, \vec v) \longmapsto t \vec v = (tv_1, tv_2, \dots, tv_n)
    \end{align*}
    \item Sum (componentwise)
    \begin{align*}
        &\mathbb{R}^n \times\mathbb{R}^n \longrightarrow \mathbb{R}^n \\
        &(\vec v, \vec w) \longmapsto \vec v + \vec w = (v_1+w_1, \dots, v_n+w_n)
    \end{align*}
\end{itemize}
$\mathbb{R}^n$ equipped with these two operations has nice proprieties! Consider a set $V$ equipped with two operations:
\begin{align*}
    *&: V \times V \longrightarrow V\\
    \odot &: \mathbb{R} \times V \longrightarrow V
\end{align*}
where this operations satisfy the following proprieties:
\begin{enumerate}
    \item $*$ is associative: $\forall \ u,v,w \in V \quad (u*v)*w = u*(v*w)$
    \item $*$ is commutative: $\forall \ u,v \in V \quad u*v = v*u$
    \item extistence of identity: $\exists e \in V$ s.t. $\forall \ v \in V \quad e*v=v$
    \item existence of inverses: $\forall \ v \in V,\ \exists w \in V$ s.t. $v*w = e$
    \item $\forall \ v \in V$, $1\odot v = v$
    \item $\forall \ s,t \in \mathbb{R}, \ \forall \ v \in V$, \quad $(s+t) \odot v = (s \odot v) * (t \odot v)$
    \item $\forall \ s \in \mathbb{R}, \forall \ v,w \in V \quad t \odot (v*w)=(t\odot v) * (t \odot w)$
\end{enumerate}
Then we say that $V$ is a vector space over $\mathbb{R}$ (real vector space)
\begin{remark}[Remark]
    $\mathbb{R}^n$ with ore operation is a vector space over $\mathbb{R}$
\end{remark}
In a vector space $V$ we always have a basis that is a set of elements such that all the elements of $V$ are linear combination of the elements of the basis.

In $\mathbb{R}^n$, a basis contains always $n$ elements, indeed a set of $n$ linear independent vectors generates all the elements of $\mathbb{R}^n$. 

In other words if $\left\{ \vec b_1, \vec b_2, \dots, \vec b_n \right\}$ is a basis of $\mathbb{R}^n$, then $\forall \ \vec v \in \mathbb{R}^n$ we have $\vec v = a_1 \vec b_1 +a_2 \vec b_2 + \dots + a_n \vec b_n$ for $a_1, \dots, a_n \in \mathbb{R}$.

A special basis is the canonical basis $\left\{ \vec e_1, \vec e_2, \dots, \vec e_n \right\}$:
$$\vec e_1 = (1,0,\dots,0) \qquad\vec e_2 = (0,1,\dots,0) \qquad \cdots \qquad \vec e_{n-1} = (0,\dots,1,0) \qquad \vec e_{n} = (0,\dots,0,1)$$
In $\mathbb{R}^n$, we say that $m$ vectors $\vec v_1, \dots, \vec v_m$ are linearly independent when
$$c_1 \vec v_1 + \dots, c_w \vec v_m = \vec 0 \iff c_1 = \dots = c_m = 0$$
We say that they are linearly dependent when there exist $c_1, \dots, c_m \in \mathbb{R}$ not all zeros such that $c_1 \vec v_1 + \dots, c_w \vec v_m = \vec 0$ 

In $\mathbb{R}^n$, we can construct a basis if we take $n$ linearly independent vectors.
\subsection{Euclidean norm}
Given a vector $\vec v \in \mathbb{R}^n$m the Euclidean norm of $\vec v$ is
\begin{align*}
    ||\vec v|| = d&(\vec v, \vec 0) = \sqrt{v^2_1 + v^2_2 + \dots + v^2_n} \\
    d&(V,O)
\end{align*}
The Euclidean norm satisfies the triangle inequality:
$$\forall \ \vec v, \vec w \in \mathbb{R}^n \qquad ||\vec v + \vec w|| \le || \vec v||  + ||\vec w||$$
\subsection{Dot product}
\begin{align*}
    \bullet : &\mathbb{R}^n \times \mathbb{R}^n \longrightarrow \mathbb{R} \\
    &\left(\vec v, \vec w\right) \longmapsto \vec v \bullet \vec w =  v_1 w_1 +  v_2 w_2 + \dots + v_n w_n 
\end{align*}
In $\mathbb{R}^2$ and $\mathbb{R}^3$ (in general in $\mathbb{R}^n$ two vectors $\vec v$ and $\vec w$ are orthogonal if and only if $\vec v \bullet \vec w = 0$
\subsection{Equation of a plane given an orthogonal vector and a point}
Considering a vector $\vec n = (n_1,n_2,n_3) \in \mathbb{R}^3$, the plane through the origin and orthogonal to $\vec n$ is the set of points $(x,y,z)$ such that the vector $(x,y,z)$ is orthogonal to $\vec n$, i.e., they must satisfy
$$(x,y,z) \bullet \left(n_1, n_2, n_3\right) = 0$$
$$\Updownarrow$$
$$n_1x+n_2y+n_3z = 0$$
Cartesian equation of the plane throught the origin and orthogonal to $\vec n$.

In general, the equation of a plane through a point $A = (x_A, y_A, z_A) \in \mathbb{R}^3$ and orthogonal to $\vec n$ in the set of points $(x,y,z) \in \mathbb{R}^3$ such that the vector $(x-x_A, y-y_A, z-z_A)$ is orthogonal to $\vec n$ and so 
$$(x-x_A, y-y_A, z-z_A) \bullet(n_1,n_2,n_3) = 0 $$
$$\Updownarrow$$
$$n_1(x-x_A) + n_2(y-y_A) + n_3(z-z_A) = 0$$
Cartesian equation of a plane through the point $A$ and orthogonal to $\vec v$.

\begin{remark}[Remark]
    Given $X \subseteq \mathbb{R}^n$, in order to check if $X$ is a vector space, it is sufficient to check if ($X \not = \emptyset$) it is closed under addition and multiplication by a scalar, i.e,
    $$\forall \ x,y \in X, \text{ check if } x+y \in X$$
    $$\forall \ x \in X, \ \forall \ \lambda \in \mathbb{R}, \text{ check if } \lambda x \in X $$
\end{remark}
\section{Matrices}
Given $m, n \in \mathbb{N} \setminus \left\{ 0 \right\}$ an $m \times n$ matrix ($m$ rows and $n$ columns). Is and object of this kind
$$A = \begin{bmatrix}
    a_{11} &a_{12} &a_{13} &\dots &a_{1n}\\
    a_{21} &a_{22} &a_{23} &\dots &a_{2n}\\
    \vdots &\vdots &\vdots &\ddots &\vdots\\
    a_{m1} &a_{m2} &a_{m3} &\dots &a_{mn}
\end{bmatrix}$$
where the elements of $A$ are $a_{ij} \in \mathbb{R} \quad \forall\ i,j \ 1\le i \le m \ \ 1 \le j \le n$.

Sometimes we will also write $\overline{A} = (a_{ij})$. We also call the elements $a_{ij}$ as element of the matrix. A matrix is an element of $\mathbb{R}^{m\times n} = \mathbb{R}^{m\cdot n} = \mathbb{R}^{m,n} $
\begin{itemize}
    \item[] If a matrix $A \in \mathbb{R}^{m\times m}$ is called a square matrix 
    \item[] A matrix in $\mathbb{R}^{1 \times n}$ is called a row vector 
    \item[] A matrix in $\mathbb{R}^{m\times 1}$ is called a column vector  
\end{itemize}
Since $\mathbb{R}^{m \times n}$ (the set of all matrices with $m$ rows and $n$ columns) is a set of the kind $\mathbb{R}^K$, we know that $\mathbb{R}^{m \times n}$ is a vector space where the operations are 
\begin{itemize}
    \item Addition (componentwise)
    $$\forall \ A,B \in \mathbb{R}^{m \times n} \qquad \text{we have } A+B=(a_{ij} + b_{ij}) \in \mathbb{R}^{m\times n}$$
    $$\begin{bmatrix}
        a_{11}+b_{11} &a_{12}+b_{12} &\dots &a_{1m}+b_{1m} \\
        \vdots &\vdots &\ddots &\vdots\\
        a_{m1}+b_{m1} &a_{m2}+b_{m2} &\dots &a_{mn}+b_{mn}
    \end{bmatrix}$$
    \item Multiplication by a scalar
    $$\forall \ A \in \mathbb{R}^{m\times n} \quad \forall \ \lambda \in \mathbb{R} \quad \text{we have } \lambda A = (\lambda a_{ij}) \in \mathbb{R}^{m \times n}$$
    $$\begin{bmatrix}
        \lambda a_{11} &\lambda a_{12} &\dots &\lambda a_{1m} \\
        \vdots &\vdots &\ddots &\vdots\\
        \lambda a_{m1} &\lambda a_{m2} &\dots &\lambda a_{mn}
    \end{bmatrix}$$
\end{itemize}
In general a linear system is a set of equation with a certain number of unknowns that must satisfy these equations simultaneously:
\[
\begin{cases}
\begin{array}{cccccc}
a_{11}x_1 &+& a_{12}x_2 &+& \cdots &+ a_{1n}x_n = b_1\\
a_{21}x_1 &+& a_{22}x_2 &+& \cdots &+ a_{2n}x_n = b_2\\
\vdots    && \vdots    && \ddots & \vdots\\
a_{m1}x_1 &+& a_{m2}x_2 &+& \cdots &+ a_{mn}x_n = b_m
\end{array}
\end{cases}
\]

where $x_1, x_2, \dots, x_n$ are unknowns 
\begin{itemize}
    \item[] $a_{ij}\in \mathbb{R}$ given real numbers, coefficients of the linear system where $1 \le i \le m$ and $1\le j \le n$
    \item[] $b_k \in \mathbb{R}$ given real numbers, constant term of the linear system where $1 \le k \le m$
\end{itemize}
Given the above linear system, we can associate to it some matrices:
\begin{itemize}
    \item[] The matrix of coefficients  $A = (a_{ij}) \in \mathbb{R}^{m\times n}$
    \item[] The vector (matrix) of unknowns $\vec x = \begin{bmatrix}
        x_1\\
        x_2\\
        \vdots\\
        x_n\\
    \end{bmatrix} \in \mathbb{R}^{n\times1}$ 
    \item[] The vector (matrix) of constant tems $\vec b = \begin{bmatrix}
        b_1\\
        b_2\\
        \vdots\\
        b_m\\
    \end{bmatrix} \in \mathbb{R}^{m\times1}$ 
\end{itemize}
Linear system with $m$ equations and $n$ unknowns.

The definition of product became, thinking a linear system with
\begin{itemize}
    \item[] matrix of coefficients $A \in \mathbb{R}^{m\times n}$
    \item[] vector of unknowns $\vec x \in \mathbb{R}^{n \times 1}$
    \item[] vector of constant terms $\vec b \in \mathbb{R}^{m \times 1}$
\end{itemize}
is equivalent to 
$$A \vec x = \vec b$$
$$\mathbb{R}^{m \times n} \times \mathbb{R}^{n \times 1} \to \mathbb{R}^{m \times 1}$$
\begin{remark}[Remark]
    The dot product between two vectors $\vec v = (v_1, \dots, v_n)$ and $\vec w = (w_1, \dots, w_n)$ can be written also
    $$\begin{bmatrix}
        v_1, \dots, v_n
    \end{bmatrix} \begin{bmatrix}
        w_1\\
        \vdots\\
        w_n
    \end{bmatrix} = v_1 w_1 + v_2 w_2 + \dots + v_n w_n = \vec v \bullet \vec w \to \mathbb{R}$$
\end{remark}
\begin{definition}
    The transpose of a matrix $A \in \mathbb{R}^{m\times n}$ is a new matrix 
    $$A^T \in \mathbb{R}^{n \times m}$$
    where the rows of $A^T$ are the columns of  $A$ or equivalently the column of $A^T$ are the rows of $A$
    $$A = \begin{bmatrix}
        1 &\pi &-\sqrt{2} &3\\
        0 &-1 &0 &3\\
        \sqrt{3} &\frac{1+\sqrt{5}}{2} &2\pi &-1
    \end{bmatrix} \quad A^T = \begin{bmatrix}
        1 &0 &\sqrt{3}\\
        \pi &-1 &\frac{1 + \sqrt{5}}{2}\\
        -\sqrt{2} &0 &2\pi\\
        3 &3 &-1
    \end{bmatrix}$$
\end{definition}
\begin{remark}[Remark]
    The product between matrices is non commutative!
\end{remark}
When $A$ and $B$ are square matrices, e.g., in $\mathbb{R}^{n\times n}$ then 
$$AB \in \mathbb{R}^{n \times n} \quad \text{and} \quad BA \in \mathbb{R}^{n \times n} \qquad \text{but usually}$$
$$AB \not = BA$$
\begin{example}
    $$A = \begin{bmatrix}
        1 &0\\
        0 & 0
    \end{bmatrix} \quad B = \begin{bmatrix}
        0 &1\\
        0 &0
    \end{bmatrix}$$
    $$AB = \begin{bmatrix}
        0 &1\\
        0 & 0
    \end{bmatrix} \quad BA =\begin{bmatrix}
        0 &0\\
        0 & 0
    \end{bmatrix}$$
\end{example}
In the example $AB = 0$ product is $0$ even if $A, B \not = 0$
$$A + \begin{bmatrix}
        0 &0\\
        0 & 0
    \end{bmatrix} = A$$
in general if $A \in \mathbb{R}^{m \times n}$, then $0  \in \mathbb{R}^{m \times n}$ whole element are all zeros is the identity.
\begin{definition}
    The matrix $I_n \in \mathbb{R}^{n \times n}$ with $I_n = \begin{bmatrix}
        1 &\cdots &0\\
        \vdots &1 &\vdots \\
        0 &\cdots & 1
    \end{bmatrix}$
    
    \begin{align*}
        I_n = \left(I_{ij}\right) \qquad &I_{ij} = 0 \ \text{ if } \ i \not = j\\
        &I_{ii} = 1 \ \text{ for all } i = 1, \dots, n 
    \end{align*}
    is the \textbf{identity matrix}. Indeed $\forall \ A \in \mathbb{R}^{n \times n}$ we have $AI_n = I_nA = A$ 
\end{definition}
\section{Solving linear systems}
\begin{definition}
    Two linear system are equivalent if they have the same solutions 
    $$\begin{cases}
        x + 2y + z = 1 \\
        3x -4y +z = 0
    \end{cases} \quad \text{is equivalent to} \quad \begin{cases}
        x+2y+z = 1 \\
        2x - y +z = \frac{1}{2}
    \end{cases} $$
\end{definition}
First of all, we focus on homogeneous linear system, which are linear system where the constant term are all zeros: 
$$A \vec x = \vec 0$$
Our goal is to introduce some «transportation» which are called elementary operations on the matrix $A$ such that the new matrix $B$ where
$$B \vec x = \vec 0 \quad \text{is equivalent to} \quad A\vec x = \vec 0$$
Let us define there elementary operations $\quad A \in \mathbb{R}^{m\times n}$
\begin{enumerate}
    \item \textbf{Row switching}: switch two rows of $A$ (if we switch two rows of $A$ this is equivalent to switch two equations of the linear system $A \vec x = \vec 0$; we get an equivalent new linear system).
    
    We can switch two rows of  $A$ performing the following product:
    $$S_{ij}A$$
    where $S_{ij}$ is a matrix in $\mathbb{R}^{m\times m}$ obtained from the identity matrix $I_m$ switching the $i^{th}$ row of $I_m$ with the $j^{th}$ row of $I_m$
    $$A \vec x = \vec 0$$
    $$\Updownarrow$$
    $$S_{ij}A \vec x = S_{ij} \vec 0$$
    $$\Updownarrow$$
    $$B \vec b = \vec 0 $$
    where $B$ is obtained from $A$ switching the $i^{th}$ row with the $j^{th}$ row
    \item \textbf{Row multiplication}: Multiply a row of $A$ by a scalar $\lambda \in \mathbb{R}$ (if you multiply an equation by  a scalar you get an equivalent equation)

    We can obtain this operation, considering the following product
    $$D_i(\lambda)A$$
    Where $D_i(\lambda)$  is a diagona matrix $m \times m$ obtained from $I_m$ replacing $I_{ii}$ with $\lambda$
    $$A \vec x = \vec 0 \iff D_i(\lambda)A \vec x = \vec 0 \iff B \vec x = \vec 0$$
    \item \textbf{Row addition}: add to a row of $A$ anther row of $A$ multiplied by a scalar $\lambda \in \mathbb{R}$. We can obtain this elementary operation, considering the following product.
    $$E_{ij} (\lambda) A$$
    Where $E_{ij}(\lambda)$ is a $m \times m$ matrix obtained from $I_m$ replacing the element $I_{ij}$ with $\lambda \in \mathbb{R}$. This product replaces the $i^{th}$ row of $A$ with the sum of $i^{th}$ row of $A$ with the $j^{th}$ row of $A$ multiplied by $\lambda$. 
    $$A\vec x = \vec 0 \iff E_{ij}(\lambda)A \vec x = \vec 0 \iff B \vec x = \vec 0$$
\end{enumerate}
\begin{definition}
    A square matrix $A \in \mathbb{R}^{n \times m}$ is inverible (w.r.t. the  products) if there exist a matrix $B \in \mathbb{R}^{n\times m}$ such that $AB = BA = I_n$

    In this case we write $B = A^{-1}$. The inverse of $A$ is denoted my $A^{-1}$ 
\end{definition}
Since when we start from a linear system with constant vector $\vec b$ we get an equivalent linear system with a new constant vector, it is useful to def with the argumented matrix $(A|b) \in \mathbb{R}^{m \times n+1}$

In general, in order to solve a linear system $A\vec x = \vec b$ we want to obtain an equivalent linear system $B \vec x = \vec c$ where 
$$(B| \vec c) \quad \text{(obtained from } (A| \vec b) \text{ using elementary operations)}$$
is in the so-callet \textbf{Row-echelon form}. 
\begin{definition}
    A matrix $M \in \mathbb{R}^{a \times b}$ is in \textbf{row-echelon form (REF)} if:
    \begin{enumerate}
        \item The leading (first nonzero) entry of each nonzero row (called a \emph{pivot}) is strictly to the right of the pivot of the row above it.
        \item All zero rows (if any) are at the bottom of the matrix.
    \end{enumerate}
\end{definition}



The rows of a matrix, denoted by rank($M$), is the number of pivot of any REF for $M$. A column of $M$ is called pivot column if it contains one pivot.

A matrix $M$ is in reduced row-echelon form RREF if 
\begin{enumerate}
    \item in in REF
    \item all the pivot are 1
    \item the only non-zero element of any pivot column is the pivot itself (REF is not unique. RREF is unique)
\end{enumerate}
Let consider $A \vec x = \vec b$ where $A \in \mathbb{R}^{m\times n}$ and focus on the augmented matrix $\left(A|\vec b\right)$, then we transform it into a matrix $\left(A'|\vec b'\right)$ that is REF and $A' \vec x = \vec b'$ is equivalent to $A \vec x = \vec b$
\begin{enumerate}
    \item If rank$\left(A' | \vec b' \right) \ne$ rank$\left( A' \right)$, then $A'\vec x=\vec b$ and $A \vec x = \vec b$ have no solutions
    $$A' = \begin{bmatrix}
        1 &-1 &2\\
        0 &5 &3\\
        0 &0 &-2\\
        0 & 0 &0
    \end{bmatrix} \quad \text{rank}(A') = 3$$
    $$\left(A' | \vec b'\right) = \left[
\begin{array}{ccc|c}
1 & -1 & 2 &0\\
0 & 5 & 3 & -1\\
0 & 0 & -2 & 2\\
0 & 0 & 0 &5
\end{array}
\right] \quad \text{rank}\left(A' | \vec b'\right) = 4 \ne \text{rank}(A')$$
$$\begin{cases}
\begin{alignedat}{4}
x &{}-{}& y &{}+{}& 2z &{}={}& 0\\
  &      & 5y&{}+{}& 3z &{}={}& -1\\
  &      &   &      &-2z &{}={}& 2\\
  &      &   &      & 0  &{}={}& 5
\end{alignedat}
\end{cases}$$
Indeed, this means that the column $\vec b'$ in the augmented matrix must contain a pivot, i.e., a non-zero element $k$ in correspondence to a zero row of $A'$ and this yelds to a fals identity of the kind 
$$0 = k \qquad k\in \mathbb{R} \qquad k \ne 0$$
\item If rank $\left( A'|\vec b' \right) =$ rank$(A') = n$ number of unknowns, them there exists a unique solution 
\begin{align*}
    &A' = \begin{bmatrix}
        1 &-1 &2\\
        0 &5 &3\\
        0 &0 &-2\\
        0 & 0 &0
    \end{bmatrix} \qquad A' \in \mathbb{R}^{4 \times 3} \Rightarrow \text{ 4 equations, 3 uknowns} \\
    &\\
    &\text{rank}(A') = 3 \\
    &\\
    &\left(A'|\vec b' \right) = \left[
\begin{array}{ccc|c}
1 & -1 & 2 &-1\\
0 & 5 & 3 & 0\\
0 & 0 & -2 & 3\\
0 & 0 & 0 &0
\end{array}
\right] \qquad \text{rank}\left( A'|\vec b' \right) = 3\\
&\\
&\begin{cases}
\begin{alignedat}{4}
x &{}-{}& y &{}+{}& 2z &{}={}& -1\\
  &      & 5y&{}+{}& 3z &{}={}& 0\\
  &      &   &      &-2z &{}={}& 3\\
  &      &   &      & 0  &{}={}& 0
\end{alignedat}
\end{cases}
\end{align*}
\item If rank$\left(A'|\vec b' \right) =$ rank$(A') < n$ ($n =$ number of unknowns) the linear system has infinitely many solutions.

Consider for example
\begin{align*}
A' = &\begin{bmatrix}
        1 &-1 &2\\
        0 &5 &3\\
        0 &0 &0\\
        0 & 0 &0
    \end{bmatrix} \quad \text{rank}(A') = 2 < 3 = n\\
    \left(A' | \vec b'\right) = &\left[
\begin{array}{ccc|c}
1 & -1 & 2 &\sqrt{2}\\
0 & 5 & 3 & -1\\
0 & 0 & 0 & 0\\
0 & 0 & 0 &0
\end{array}
\right] \quad \text{rank}\left(A' | \vec b'\right) = 2\\
& \qquad \Updownarrow \\
&\begin{cases}
\begin{alignedat}{4}
x &{}-{}& y &{}+{}& 2z &{}={}& \sqrt{2}\\
  &      & 5y&{}+{}& 3z &{}={}& -1\\
  &      &   &      &0 &{}={}& 0\\
  &      &   &      & 0  &{}={}& 0
\end{alignedat}
\end{cases} \qquad \forall \ z \in \mathbb{R}
\end{align*}
we have infinitely many solutions, because $z$ is a free variable, it can assume any value in $\mathbb{R}$
\end{enumerate}
\begin{theorem}[Rouché–Capelli Theorem]
    Let $A\vec x = \vec b$ be a linear system, with $A \in \mathbb{R}^{m \times n}$ $\binom{n \text{ equations}}{m \text{ unknowns}}$ and let $\left( A'|\vec b' \right)$ be in REF obrained from $\left(A | \vec b\right)$, then (this $A'\vec x' = \vec b' $ is equivalent to $A \vec x = \vec b$)
    \begin{enumerate}
        \item if rank $\left(A | \vec b\right) \ne$ rank$\left(A'\right)$, then $A'\vec x = \vec b$ doesn't have solutions 
        \item if rank $\left(A | \vec b\right) =$ rank$\left(A'\right) = n$ then $A'\vec x = \vec b$ has a unique solution
        \item if rank $\left(A | \vec b\right) =$ rank$\left(A'\right) < n$ then $A'\vec x = \vec b$ has infinitely many solutions and $n$-rank$(A')$ is the number of free variables 
    \end{enumerate}
\end{theorem}

\section{Vector subspaces}
Given a vector space $V$ and $W \subseteq V$, we say that $W$ is a vector subspace of $V$ if it is a vector space.

Given a vector space $V$ and $W \subseteq V$ then $W$ is a vector subspace if and only if
\begin{itemize}
    \item[a)] said $e \in V$ the identity w.r.t. the «addition», we have $e \in W$
    \item[b)] $\forall \ w_1, w_2 \in W$, we have $w_1 * w_2 \in W$
    \item[c)] $\forall \ w \in W$, $\forall \ \lambda \in \mathbb{R}$, we have $\lambda \odot w \in W$ 
\end{itemize}
Given a homogeneous linear system, the set of solutions (if it is not the empty set) is a vector space (in particular is a vector subspace of $\mathbb{R}^n$ where $n$ is the number of unknowns)

Consider a linear system $A \vec x = \vec 0$ where $A \in \mathbb{R}^{m \times n}$, let $S$ be the set of solutions. We can have that $S = \left\{ (0, \dots, 0) \in \mathbb{R}^n \right\} \subseteq \mathbb{R}^n$ and it is a vector space (the trivial one). In general we have that $S = \left\{ \begin{bmatrix}
    x_1\\
    \vdots\\
    x_n
\end{bmatrix} \in \mathbb{R}^n = \mathbb{R}^{n \times 1} : A \vec x = \vec 0  \right\}$
\begin{itemize}
    \item[a)] We can check tat $\vec 0 \in S$
    \item[b)] $\forall \ \vec v, \vec w \in S$, i.e., $A \vec v = \vec 0$ and $A \vec w = \vec 0$ we want to check that $\vec v + \vec w \in S$, i.e., we want to check that 
    $$A\left(\vec v + \vec w\right) = \vec 0$$
    indeed we can observe that $A \left( \vec v + \vec w \right) = A \vec v + A \vec w = \vec 0 + \vec 0 = \vec 0$
    \item[c)] $\forall \ \vec v \in S$, $\forall \ \lambda \in \mathbb{R}$ we check that $\lambda \vec v \in S$, i.e., $A \left(\lambda \vec v \right) = \vec 0$ 
\end{itemize}
Given a non-homogeneous linear system, the set of its solutions is not a vector space.

Let us consider $A \vec x = \vec b$ where $A \in \mathbb{R}^{m \times n}$ and $\vec b \ne \vec 0$ then $\vec 0 \not \in S =\left\{ \begin{bmatrix}
    x_1\\
    \vdots\\
    x_n
\end{bmatrix} \in \mathbb{R}^n  : A \vec x = \vec b \right\}$ because $A \vec 0 = \vec 0 \ne \vec b$
\begin{definition}
    Let $V$ be a vector space and $v_1, v_2, \dots, v_k \in V$ we define the set generated by $v_1, \dots, v_k$, and we call it the spanning set of $v_1, \dots, v_n$ the following set
    $$\text{span}(v_1, \dots, v_k) = \left< v_1, \dots, v_k \right> = \left\{ c_1 v_1 + \cdots + c_k v_k : \forall \ c_1, \dots, c_k \in \mathbb{R} \right\}$$
\end{definition}
\begin{example}
    In $\mathbb{R}^3$, given two vectors $\vec v_1, \vec v_2$ linearly independent, their spamming set is a plane through the origin
    $$\text{span}(\vec v_1, \vec v_2) = \left\{ s\vec v_1 + t \vec v_2 : \forall \ s,t \in \mathbb{R} \right\}$$
\end{example}
\begin{definition}
    Let $V$ be a vector space, we say that $\left\{ v_1, \dots, v_k \right\}$ with $v1, \dots, v_k \in V$ is set of generators of $V$ if $V = \text{span}(v_1, \dots, v_k)$
\end{definition}
\begin{definition}
    Let $V$ be a vector space, we say that $\left\{ b_1, \dots, b_n \right\}$ with $b_1, \dots, b_n \in V$ is a basis of $V$ if
    \begin{itemize}
        \item $V = \text{span}(b_1, \dots, d_n)$
        \item $b_1, \dots, b_n$ are linearly independent
    \end{itemize}
\end{definition}
\begin{definition}
    Let $V$ be a vector space, we sat that $w_1, \dots, w_h \in V$ are linearly independent if (given $c_1, \dots, c_h \in \mathbb{R}$)
    $$c_1 w_1 + \dots + c_h w_h = 0 \quad (1)$$
    $$\Updownarrow$$
    $$c_1 = \dots = c_h = 0$$
    if $(1)$ has a solution $c_1, \dots, c_h$ not all zeros, then $w_1, \dots, w_h$ are linearly dependent.
\end{definition}

\end{document}
