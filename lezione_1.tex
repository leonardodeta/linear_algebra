\section{Introduction}
In linear algebra we study object, also a point is a linear object. Usually we study this object from an geometry point of view, in this course we see the algebrian point of view. When we do \textbf{operations} we are doing algebra. Vector are a kind of linear object, we can see a vector like a point.

In $\mathbb{R}^3$ we have more kind of linear objects. In $\mathbb{R}^4$ we start to have problem whit the geometrical point of view. Whit equations i can describe objects in every dimension. Vector spaces is very important in linear algebra. Matrix are very important because we also have functions for transform object.
\section{Sets}
We denote by $\emptyset$ the empty set.
\begin{align*}
&\mathbb{N} = \left\{ 0, 1, 2, \cdots \right\} \text{ Natural numbers}\\
&\mathbb{Z} = \left\{\cdots, -2, -1, 0, 1, 2, \cdots \right\} \text{ Integers}\\
&\mathbb{Q} = \left\{ \frac{a}{b} : a, b \in \mathbb{Z}, b \ne 0 \right\}
\end{align*}
\begin{center}
    $\mathbb{R}$ the set of all  real numbers, Every real number is the limit of a sequence of rational numbers.
\end{center}

\subsection{Examples}
\begin{align*}
&A = \left\{ x \in \mathbb{R} : -5 \le x \le 2\right\}11
&B = \left\{ x \in \mathbb{Z} : -5 \le x \le 2 \right\}\\
&C = \left\{ x \in \mathbb{R} : x > 1 \right\}
\end{align*}
\subsection{Operation between sets}
Given two set $A$ and $B$ we define the following operation: 
\begin{itemize}
    \item Intersection: $A \cap B = \left\{ x : x \in A \cap x \in B\right\}$
    \item Union $A \cup B = \left\{ x: x \in A \lor X \in B \right\}$
    \item Difference $A \setminus B = \left\{ x : x \in A \land x \not\in B \right\}$
    \item Cartesian product $A \times B = \left\{ (x,y): x \in A \land y \in B \right\}$
\end{itemize}
\section{Cartesian plane}
The cartesian plane is a 2 dimensional object. All the $\mathbb{R}$ can be represented on an «oriented line» where we fixed a special point, namly the origin O. In a similar way we can describe all the elements of $\mathbb{R} \times \mathbb{R} = \mathbb{R}^2 = \left\{ (x, y) : x \in \mathbb{R}, y \in \mathbb{R}\right\}$ with the cartesian plane. We take 2 oriented lines which are orthogonal and we call the origin O the interaction between these lines (representing the element $(0, 0) \in \mathbb{R}^2$.

The horizontal line is called the axis of abscises or the x-axis. The vertical line is called of ordinate or the y-axis.

Evrey point $P$ descibe an element of $\mathbb{R}^2$ since it is identified by two set number $X_P$ and $X_P$ which are its cordinated, i. e., 

\begin{align*}
    \text{The y-axis is the following set } &= \left\{ (0,y) \in \mathbb{R}^2 \right\} = \left\{ (0,y): y \in \mathbb{R}\right\} \\
    & = \left\{ (x,y) \in \mathbb{R}^2 : x = 0 \land y \in \mathbb{R} \right\}
\end{align*}
\begin{align*}
    \text{The x-axis is the following set } &= \left\{ (x,0) \in \mathbb{R}^2 \right\} 
    = \left\{ (z,0): z \in \mathbb{R}\right\} \\
    &= \left\{ (x_1,x_2) \in \mathbb{R}^2 : x_2 = 0 \right\}
\end{align*}

Given two point $A$ and $B$ $\in \mathbb{R}^2$, we can evaluate theri distance (euclidean distance) which is the lengh of the segment $\overline{AB}$:
\begin{align*}
    d(A,B)=||\overline{AB}|| &= \sqrt{(x_B - x_A)^2 + (y_B- y_A)^2} \\
    &= \sqrt{(x_A - x_B)^2 + (y_A- y_B)^2}
\end{align*}
\subsection{Vector in the cartesian plane}
In general a vector is an arrow. Given a point $P$ in the cartesian plane, i. e., a point of two numbers $(x_P, y_P) \in \mathbb{R}^2$ it also indentified a vector of the plane. 
The point $P$ identifies the vector $\vec{OP} = (x_P, y_p)$. Sometimes we will also use the notion $\vec{\mathbf{v}}$ or $\mathbf{v}$ for a vector.
$$\mathbf{v} = (\mathbf{v}_1, \mathbf{v}_2) \in \mathbb{R}^2$$
We have that, from this algebric point of view, point and vector are equivalent, they are the same object described by a point or set. 
\subsection{Operations in the cartesian plane}
\subsubsection{Scalar multiplication}
Multiplication by a sclar, i. e., by a real number. Givern a point $A = (x_A, y_A)(\Leftrightarrow Q = (x_A, y_A) = \vec{OA})$ and a scalr $t\in \mathbb{R}$ we define the multiplication by a scalar $tA = (tx_A, ty_A)$. 
$$\mathbb{R}^2 \times\mathbb{R} \longrightarrow \mathbb{R}^2$$
$$(A\times t) \longrightarrow tA$$

The effect of the multiplication by a scalar on a vector $\vec{\textbf{a}}$ ub fact the vertor $\mathbb{a}$ is stretched

\subsubsection{Sum}
Given two points $A = (x_A, y_A)( \Leftrightarrow a = (x_A, y_A))$ and $B= (x_B,y_B)$ the sum is defined as
$$A+B = (x_A+x_B, y_A+y_B)$$
\begin{align*}
\mathbb{R}^2 \times \mathbb{R}^2 &\longrightarrow \mathbb{R}^2\\
(A,B) &\longrightarrow A+B
\end{align*}

$a + b (\Leftrightarrow A+B)$ is the diagonal of the parallelogram whith $a$ and $b$ 

the following proprieties hold:
$$\forall \ a, b, c \in \mathbb{R}^2, \forall \ t \in \mathbb{R} \text{ we have}$$
\begin{itemize}
    \item $a+b = b+a$
    \item $(a+b) + c = a + (b+ c)$
    \item $t(a+b)=ta + ta$
    \item $a + 0 = a$ when $0 = (0,0) =$ origin
    \item $(-1)a = -a$ when $a+(-a)=0$
\end{itemize}
\subsection{Lines in the cartesian plane}
Given a point $A = (x_A, y_A) \in \mathbb{R}^2$, there exist one and only one line $\ell$ through the origin O and the point A. Now we went to describe the coordinates of all point. 

Let us take a generic point $P=(x_P, y_P)\in \ell$ and let us find its coordinate in term of $A$. The triangles $OAH$ and $OPC$ are similar (they have the same angles) and consequently the side are proportion (by the same factors):
$$||\overline{OK}|| = t||\overline{OH}|| \quad \text{and} \quad  ||\overline{PK}|| = t ||\overline{AH}|| \text{ for a certain } t \in \mathbb{R}$$
that is
$$\begin{cases}
    x_p' =t x_A\\
    y_p' = ty_A
\end{cases} \quad \forall \ t \in \mathbb{R} \qquad \begin{cases}
    x = tx_a\\
    y= ty_A
\end{cases} \quad \forall \ t \in \mathbb{R}$$
These are the parametric equations of a line through the origin and given a point $A$. 
$$A = (2,3) \qquad \begin{cases}
     x= 2t \\
    y = 3t
\end{cases} \quad \forall \ t \in\mathbb{R}$$
$$\text{In other words} \quad \ell = \left\{ (x,y) \in \mathbb{R}^2: x=tx_A, y=ty_A, \forall \ t \in \mathbb{R} \right\} \quad (1)$$
The cartesian equation of a line through the origin is 
$$y = mx \qquad \text{where} \qquad m = \frac{y_A}{x_A} \quad \text{slope}$$
$$\ell \left\{ (x,y) \in \mathbb{R}^2 : y = mx, m=\frac{y_A}{x_A} \right\} \quad (2)$$
Let us Considering the set $(1)$ and check that $O$ ans $A$ belong to this set:
$$O \in \ell \text{ because if we take } t=0, \text{ we get } x = 0 \cdot X_A= 0, y = 0\cdot y_A = 0 
$$
$$\Rightarrow (0,0) \in \ell$$
$A \in \ell$ because if we take $t=1$, we have $x=x_A$, $y=y_A \Rightarrow A \in \ell$

Let us check that $O$ and $A$ belong to $   \ell$ exploiting the set defined in $(2)$:
\begin{align*}
    &O = (0,0) \in \ell \quad\text{ is true, because } \quad 0 = m\cdot 0\\
    &A= (x_A, y_A) \in \ell \quad \text{ is true, because } \quad y_A = mx_A \Leftrightarrow y_A = \frac{y_A}{\not{ x_A}} \cdot \not x_A
\end{align*}
We can check that a point provided by the parametric equations satisfies the cartesian equation. 

Let us consider a point $(tx_A, ty_A)$ and check if it satisfies the cartesian equation $y=mx$:
$$ty_A = mtx_A \Leftrightarrow ty_A = \frac{y_A}{\not x_A}t \not x_a \Leftrightarrow ty_A = ty_Ax$$
\begin{example}
    Consider the line through the origin and $A = (5,2)$.
    $$\text{The parametric equation are } \quad \begin{cases}
        x= 5t\\
        y= 2t
    \end{cases} \quad \forall t \in \mathbb{R} $$
    $$t = 0 \quad (0,0), \qquad t= 1 \quad (5,2), \qquad t=-2 \quad \left(-10,-4\right),  $$ 
    $$t= \frac{1}{3} \quad \left(\frac{5}{3}, \frac{2}{3}\right) \qquad \dots \qquad t=\sqrt{3} \quad \left(5\sqrt{3},2\sqrt{3}\right)$$
    All belong to the line. 
    
    The point $(10,3)$ belong to the line? $ \qquad t=2 \quad (10,4)$

    In order to answer to this question we must solve the following linear system
    $$\begin{cases}
        5t = 10\\
        2t = 3
    \end{cases}$$
    where $t$ is the unknown, but this linear system doesn't have solutions!

    This $(10,3)$ doesn't belong to the line
    The cartesian equation of this line is $y = \frac{2}{5}x \to 3=\frac{2}{5}\cdot10$ is not true! is not an identity
\end{example}
Now, let us describe the parametric equations of a line (not necessary through the origin). Given two points $A=(x_A, y_A)$ and $B = (x_B, y_B)$ we want to describe the coordinates of a generic point $P$ over the line through $A$ and $B$. We can consider the line $\ell'$ parallel to $\ell$ and passing through the origin. The parametric equations of $\ell'$ are:
$$\begin{cases}
    x = t(x_B-x_A)\\
    y = t(y_B-y_A)
\end{cases} \quad \forall \ t \in \mathbb{R}$$
The parametric equations of $\ell$ are
$$\begin{cases}
    x = x_A + t(x_B - x_A)\\
    y = y_A + t(y_B - y_A)
\end{cases} \quad \forall \ t \in \mathbb{R}$$
this 
$$\ell \left\{ (x,y) \in \mathbb{R}^2 : x = x_A + t (x_B-x_A), y = y_A + t(y_B-y_A),  \ \forall \ t \in \mathbb{R} \right\}$$
Similarly, tha cartesian equation is
$$y = mx + q$$
where $m = \frac{y_B - y_A}{x_B - x_A}$ slope and $q = \frac{y_A x_B - x_Ay_B}{x_B-x_A}$ constant term
